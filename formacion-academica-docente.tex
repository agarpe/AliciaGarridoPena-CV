\begin{rubric}{Formación Académica y Docente}
	\subrubric{Cursos de Formación Docente}
	
	\entry*[9-11 de mayo de 2022] \textbf{Hacia una Universidad Inclusiva: Elementos Transformadores (6h)}, Programa de Formación Docente. Universidad Autónoma de Madrid (UAM).
	
	
	\entry*[15-17 de febrero de 2023] \textbf{Cómo Evaluar los Resultados de Aprendizaje}, Programa de Formación Docente. Universidad Autónoma de Madrid (UAM). \textbf{Créditos}: 1 ECTS.
	
	\entry*[20-27 de abril de 2023] \textbf{Detección y Manejo de Problemas Psicológicos entre Estudiantes}, Programa de Formación Docente. Universidad Autónoma de Madrid (UAM). \textbf{Créditos}: 1 ECTS.
	
	\entry*[17-24 de octubre de 2023] \textbf{El Concepto de Competencia y la Formación Basada en Competencias}, Programa de Formación Docente. Universidad Autónoma de Madrid (UAM). \textbf{Créditos}: 1 ECTS.
	
	\entry*[19-20 de febrero de 2024] \textbf{Curso de Autoridad y Liderazgo (1 ECTS)}, Universidad Autónoma de Madrid (UAM). Programa de Formación Docente.
	
	
	\subrubric{Cursos de Investigación}
	
	\entry*[Febrero 2024] \textbf{Curso de Postprocesamiento de fMRI: Estado en Reposo y FSL}, organizado por el Centro de Neurociencia Clínica. Hospital Los Madroños. \textbf{Horas}: 12 horas.
	
	\entry*[Octubre 2021] \textbf{Hackathon Brain Code Games}, organizado por la Sociedad Española de Neurociencia (SENC) y la Universidad Autónoma de Madrid (UAM). \textbf{Horas}: 60 horas.
	
	\entry*[Octubre 2020] \textbf{Primera Escuela de Verano Neutouch: Tacto para Prótesis}, en línea. Organizador: Neutouch.
	
	\entry*[Abril 2020] \textbf{Escuela de Primavera BCI y Neurotecnología 2020}, en línea. Organizador: g.tec.
	
	
	\entry*[18 de junio de 2019] \textbf{Riesgo Biológico (4h)}, Madrid. Organizador: Servicio de Prevención de Riesgos, Universidad Autónoma de Madrid (UAM).
	
	\entry*[4 de julio de 2019] \textbf{Seguridad en el Manejo de Productos Químicos (4h)}, Madrid. Organizador: Servicio de Prevención de Riesgos, Universidad Autónoma de Madrid (UAM).
	
	
	\subrubric{Cursos y actividades de Divulgación Científica e Interdisciplinariedad}
	
	\entry*[4 y 11 de marzo de 2024] \textbf{Organización e Impartición del curso titulado: "Software Libre para la Carrera Investigadora"}, Universidad Autónoma de Madrid (UAM).
	
	\entry*[25 de enero de 2023] \textbf{Conferencia "Ciencia Abierta en la Universidad Española y su Impacto en el Desarrollo y Evaluación de la Investigación"}, Instituto Interuniversitario INAECU.
	
	\entry*[Mayo 2021] \textbf{Concurso de Divulgación \#HiloTesis}, organizado por CRUE.
	
	\entry*[Octubre 2020 a febrero 2021] \textbf{Vistas Interdisciplinarias en la Investigación Científica (5 ECTS)}, Madrid. Organizador: EDUAM, Universidad Autónoma de Madrid (UAM).
	
	\entry*[Febrero a junio 2018] \textbf{Curso de Animador Científico en Museo de Ciencias (100h; 4 ECTS)}, Granada. Organizador: Parque de las Ciencias; Universidad de Granada (UGR).
	
	\subrubric{Cursos y actividades Mujeres en Ciencia y Tecnología}
	
	\entry*[Junio 2024] \textbf{Taller de IA para el campus "Quiero ser Ingeniera"}, Universidad Autónoma de Madrid (UAM).
	
	\entry*[Febrero 2024] \textbf{Curso de Comunicación Científica con Perspectiva de Género}, Unidad de Cultura Científica, Universidad Autónoma de Madrid (UAM).
	
	\entry*[Junio 2022] \textbf{Participación en la Mesa Redonda "Estudiar STEM"}, Proyecto: "Quiero ser Ingeniera - UAM", organizado por la Escuela de Ingeniería. Universidad Autónoma de Madrid (UAM).
	
	
	\subrubric{Otros cursos y actividades}
	
	\entry*[Febrero 2021 - junio 2021] \textbf{Organización y Participación en Seminarios del Grupo de Investigación}, organizado por GNB.
	
	\entry*[Junio 2023] \textbf{Participación en la Exposición de Pósters durante la Semana Doctoral 2023}, organizada por la Escuela de Doctorado. UAM.
	
	\entry*[Septiembre 2023] \textbf{Asistencia al Congreso Internacional IBRO}, organizado por la International Brain Research Organization.
	
	\entry*[Periodo Académico 2017/2018] \textbf{Biblioteca de la Universidad de Granada: Acciones Formativas (3 ECTS)}, Granada. Organizador: Universidad de Granada (UGR).
	
\end{rubric}